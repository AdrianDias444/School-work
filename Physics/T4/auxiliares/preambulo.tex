%%%%%%%%%%%%%%%%%%%%%%%%%%%%%%%%%%%%%%%%%%%%%%%%%%%%%%%%%%%%
%                        COMPILAÇÃO
%%%%%%%%%%%%%%%%%%%%%%%%%%%%%%%%%%%%%%%%%%%%%%%%%%%%%%%%%%%%
% Compilar sempre com: XeLaTeX + Biber
%%%%%%%%%%%%%%%%%%%%%%%%%%%%%%%%%%%%%%%%%%%%%%%%%%%%%%%%%%%%

%%%%%%%%%%%%%%%%%%%%%%%%%%%%
% Cores - PRIMEIRO carregar xcolor
%%%%%%%%%%%%%%%%%%%%%%%%%%%%
\usepackage[xetex,dvipsnames,svgnames,table]{xcolor}

% Cores definidas por CMYK (do pre.tex)
\definecolor{fondpaille}{cmyk}{0, 0.0118, 0.1176, 0, 0.01} %cmyk(0%, 2.75%, 23.53%, 0%, 1%)
\definecolor{ferrugem}{cmyk}{0, 0.4457, 0.75, 0.6392}

% Cores definidas por hexadecimal (do pre.tex)
\definecolor{moleskin}{HTML}{FFF8DC}
\definecolor{laranja}{HTML}{F07241}
\definecolor{marron}{HTML}{800000}
\definecolor{lovely}{HTML}{C04848}
\definecolor{click}{HTML}{480048}
\definecolor{tinta-velha}{HTML}{4e4643}
\definecolor{Browned_Cherry}{HTML}{910831}
\definecolor{azul}{HTML}{000080}

% Cores da capa
\definecolor{BLACK}{HTML}{000000}
\definecolor{ELECTRICRED}{HTML}{DD0000}
\definecolor{TANGERINYELLOW}{HTML}{FFCE00}

% Cor consolidada para texto principal
\colorlet{cor-da-tinta}{ferrugem}
\newif\ifferrugem
\ferrugemfalse

% Cor de fundo da página - ESTA LINHA TORNA O FUNDO BEGE!
\pagecolor{fondpaille}

%%%%%%%%%%%%%%%%%%%%%%%%%%%%
% Língua e Tipografia
%%%%%%%%%%%%%%%%%%%%%%%%%%%%
\usepackage{fontspec}          % Suporte a Unicode + fontes OpenType
\usepackage{polyglossia}       % Idioma com XeLaTeX
\setdefaultlanguage{portuges}  % Português (variante europeia antiga)

% Exemplos de selecção de fontes:
% \setmainfont{EBGaramond}[ ... ]
% \setmainfont{TeX Gyre Pagella}[ ... ]

%%%%%%%%%%%%%%%%%%%%%%%%%%%%
% Estética de Parágrafos / Quebras
%%%%%%%%%%%%%%%%%%%%%%%%%%%%
\clubpenalty=100
\linepenalty=100
\widowpenalty=100
\linespread{1.3}

% Melhorar justificação e reduzir underfull hboxes
\tolerance=2000
\emergencystretch=2em
\hbadness=5000

%%%%%%%%%%%%%%%%%%%%%%%%%%%%
% URL e Quebras
%%%%%%%%%%%%%%%%%%%%%%%%%%%%
\usepackage{xurl}
\urlstyle{same}

\AtBeginDocument{%
  \setlength{\Urlmuskip}{0mu plus 1mu}
  \setlength{\emergencystretch}{3em}
  \def\UrlBreakPenalty{100}
  \def\UrlBigBreakPenalty{50}
  \sloppy
}

% Novo estilo para URLs
\makeatletter
\def\url@luisstyle{%
  \@ifundefined{selectfont}{\def\UrlFont{\sf}}{\def\UrlFont{\small\ttfamily}}}
\makeatother
\urlstyle{luis}

%%%%%%%%%%%%%%%%%%%%%%%%%%%%
% Legendas
%%%%%%%%%%%%%%%%%%%%%%%%%%%%
\usepackage[hang,small,bf]{caption}
\setlength{\captionmargin}{1.5in}

% Subfiguras (compatível com caption e XeLaTeX)
\usepackage{subcaption}

%%%%%%%%%%%%%%%%%%%%%%%%%%%%
% Geometria da Página
%%%%%%%%%%%%%%%%%%%%%%%%%%%%
\usepackage[
  a4paper,
  left=2.5cm, right=2.5cm,
  top=2.5cm, bottom=2.5cm,
  includehead, includefoot
]{geometry}

%%%%%%%%%%%%%%%%%%%%%%%%%%%%
% Bibliografia (CORRIGIDO)
%%%%%%%%%%%%%%%%%%%%%%%%%%%%
\usepackage[
    backend=biber,
    style=numeric-comp,
    language=auto,
    natbib=true,  % Alterado para true para compatibilidade com \cite
    block=space,
    isbn=false,
    url=true,
    doi=true,
    mcite=true,
    eprint=false,
    sorting=none,
    maxnames=6
]{biblatex}
\addbibresource{bibliografia.bib}
%%%%%%%%%%%%%%%%%%%%%%%%%%%%
% Hyperref (usar após quase tudo)
%%%%%%%%%%%%%%%%%%%%%%%%%%%%
\usepackage[
  xetex,
  unicode=true,
  bookmarks=true,
  bookmarksopen=true,
  bookmarksnumbered=true,
  linktoc=all,
  breaklinks=true,
  colorlinks=true,
  citecolor=black,  % Alterado para preto para combinar com ferrugem
  filecolor=black,
  linkcolor=ferrugem,  % Links na cor ferrugem
  urlcolor=azul,       % URLs em azul (definido acima)
  pdfpagelabels=true,
  plainpages=false,
  pdftitle={Desafios 12 ano},
  pdfauthor={Desafios 12 ano},
  pdfsubject={Desafios 12 ano},
  pdfkeywords={Desafios 12 ano},
  pdfcreator={XeLaTeX com hyperref},
  pdfproducer={XeLaTeX}
]{hyperref}

%%%%%%%%%%%%%%%%%%%%%%%%%%%%
% Notas de rodapé
%%%%%%%%%%%%%%%%%%%%%%%%%%%%
\usepackage{fancyhdr}

\renewcommand{\thefootnote}{\alph{footnote}}
\makeatletter
\renewcommand\@makefnmark{\@textsuperscript{\normalfont(\@thefnmark)}}
\makeatother

\long\def\symbolfootnote[#1]#2{%
  \begingroup
  \def\thefootnote{\fnsymbol{footnote}}%
  \footnote[#1]{#2}%
  \endgroup
}

%%%%%%%%%%%%%%%%%%%%%%%%%%%%
% Caixas e estilo
%%%%%%%%%%%%%%%%%%%%%%%%%%%%
\renewcommand{\fboxrule}{2pt}

%%%%%%%%%%%%%%%%%%%%%%%%%%%%
% Estilo das listas
%%%%%%%%%%%%%%%%%%%%%%%%%%%%
\renewcommand{\labelenumi}{\textbf{\alph{enumi}.}}
\renewcommand{\labelenumii}{\textbf{(\arabic{enumii})}}

%%%%%%%%%%%%%%%%%%%%%%%%%%%%
% Espaçamento (opcional)
%%%%%%%%%%%%%%%%%%%%%%%%%%%%
\usepackage{setspace}

%%%%%%%%%%%%%%%%%%%%%%%%%%%%
% Pacotes gerais
%%%%%%%%%%%%%%%%%%%%%%%%%%%%
\usepackage{comment}
\usepackage{siunitx}  % PRIMEIRO carregue siunitx
\usepackage{relsize}

% Comando CC++
\newcommand\CC{C\nolinebreak[4]\hspace{-.05em}\raisebox{.4ex}{\relsize{-3}{\textbf{++}}}}

%%%%%%%%%%%%%%%%%%%%%%%%%%%%
% Listagens de código
%%%%%%%%%%%%%%%%%%%%%%%%%%%%
\usepackage{listings}
\usepackage{lstlinebgrd}

% Definir cores personalizadas
\definecolor{lightgray}{gray}{0.95}
\definecolor{darkgray}{gray}{0.4}
\definecolor{purple}{rgb}{0.58,0,0.82}

% Traduzir o título "Listing" -> "Listagem"
\renewcommand{\lstlistingname}{Listagem}
\renewcommand{\lstlistlistingname}{Lista de Listagens}

% Definição do estilo CORRIGIDA - sem LightYellow1
\lstset{
  basicstyle=\ttfamily\small,
  keywordstyle=\color{blue},
  stringstyle=\color{red!70!black},
  commentstyle=\color{green!50!black},
  breaklines=true,
  showstringspaces=false,
  upquote=true,
  extendedchars=true,
  literate={~}{{\textasciitilde}}1,
  numbers=left,
  numberstyle=\tiny\color{darkgray},
  numbersep=8pt,
  backgroundcolor=\color{lightgray},
  frame=single,
  rulecolor=\color{black!30},
  language=python,
  captionpos=t,
  xleftmargin=2em,
  aboveskip=1em,
  belowskip=1em
}

\usepackage{booktabs}

%%%%%%%%%%%%%%%%%%%%%%%%%%%%
% TikZ e gráficos
%%%%%%%%%%%%%%%%%%%%%%%%%%%%
\usepackage{pgf,tikz}
\usepackage{tikzpagenodes}
\usetikzlibrary{backgrounds,calc,positioning,shapes}

\pgfdeclarelayer{background}
\pgfdeclarelayer{foreground}
\pgfsetlayers{background,main,foreground}

%%%%%%%%%%%%%%%%%%%%%%%%%%%%
% PythonTeX
%%%%%%%%%%%%%%%%%%%%%%%%%%%%
\usepackage{pythontex}

%%%%%%%%%%%%%%%%%%%%%%%%%%%%
% Citações
%%%%%%%%%%%%%%%%%%%%%%%%%%%%
\usepackage{csquotes}
