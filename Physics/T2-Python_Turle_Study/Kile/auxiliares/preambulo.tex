%\usepackage{xunicode} % https://ctan.org/pkg/xunicode
%\usepackage{xltxtra} % https://ctan.org/pkg/xltxtra
\usepackage{fontspec} % https://ctan.org/pkg/fontspec
\usepackage{polyglossia} % https://ctan.org/pkg/polyglossia
\setdefaultlanguage{portuges}
%\setdefaultlanguage{portuguese}
%\setdefaultlanguage[spelling=old]{portuguese}
\usepackage{csquotes}

%\usepackage{ebgaramond}%[RawFeature={+onum,+pnum,+ss06}]
\usepackage[cmintegrals,cmbraces]{newtxmath}
%\usepackage{ebgaramond-maths}

\usepackage[no-math]{fontspec}
%\setmainfont{EBGaramond}[
%  Extension=.otf,
%  UprightFont=*-Regular,
%  ItalicFont=*-Italic,
%  BoldFont=*-Bold,
%  BoldItalicFont=*-BoldItalic,
%  Numbers={Proportional,OldStyle},
%  Ligatures=TeX
%]

%\setmainfont{TeX Gyre Pagella}[
%  Extension = .otf,
%  UprightFont = *-Regular,
%  ItalicFont = *-Italic,
%  BoldFont = *-Bold,
%  BoldItalicFont = *-BoldItalic,
%  Ligatures = TeX
%]



\clubpenalty=100
\linepenalty=100
\widowpenalty=100

%estética
% \usepackage[hyphens]{url}
\usepackage{xurl}
\urlstyle{same} % URLs na mesma fonte do texto (Times New Roman)
\AtBeginDocument{%
  \setlength{\Urlmuskip}{0mu plus 1mu} % Ajusta espaçamento entre caracteres em URLs
  \setlength{\emergencystretch}{3em} % Permite maior flexibilidade na justificação
  \def\UrlBreakPenalty{100} % Penalidade baixa para facilitar quebras
  \def\UrlBigBreakPenalty{50} % Penalidade para quebras em caracteres como /
  \sloppy % Relaxa a justificação para evitar ultrapassagem
}

\usepackage[hang,small,bf]{caption}
\usepackage[a4paper,left=2.5cm,right=2.5cm,top=2.5cm,bottom=2.5cm,includehead,includefoot]{geometry}
\linespread{1.3}

\usepackage[
  xetex,                      % Driver para XeLaTeX
  unicode=true,               % Suporte a caracteres Unicode (ex.: ç, ã)
  bookmarks=true,             % Cria bookmarks no PDF
  bookmarksopen=true,         % Bookmarks abertos por padrão
  bookmarksnumbered=true,     % Numera os bookmarks (ex.: 1, 1.1, 1.2)
  linktoc=all,                % Links clicáveis em sumário, figuras e citações
  breaklinks=true,            % Permite quebra de links longos
  colorlinks=true,            % Links com cor em vez de caixas
  citecolor=black,            % Cor das citações (preto)
  filecolor=black,            % Cor dos links de arquivos (preto)
  linkcolor=black,            % Cor dos links internos (preto)
  urlcolor=blue,              % URLs em azul
  pdfpagelabels=true,         % Adiciona rótulos às páginas do PDF
  plainpages=false,           % Diferencia numeração de páginas
  pdftitle={Desafios 12 ano},
  pdfauthor={Desafios 12 ano}, % Nome do autor
  pdfsubject={Desafios 12 ano},
  pdfkeywords={Desafios 12 ano},
  pdfcreator={XeLaTeX com hyperref},
  pdfproducer={XeLaTeX}
]{hyperref}

%% Definir um novo estilo de url
\makeatletter
\def\url@luisstyle{%
  \@ifundefined{selectfont}{\def\UrlFont{\sf}}{\def\UrlFont{\small\ttfamily}}}
\makeatother
\urlstyle{luis}

\setlength{\captionmargin}{1.5in}
\usepackage{fancyhdr}

\renewcommand{\thefootnote}{\alph{footnote}} %modificar as letras de notas de rodapé%
\makeatletter
\renewcommand\@makefnmark{\@textsuperscript{\normalfont(\@thefnmark)}}
\makeatother
\renewcommand{\fboxrule}{2pt} %para alterar a expessura das caixas
%\renewcommand{\fboxsep}{3ex} %para alterar a separação nas caixas

%para alterar a fonte dos items http://cnlart.web.cern.ch/cnlart/214/node68.html
%\renewcommand{\labelenumi}{\textbf{\arabic{enumi}.}}
\renewcommand{\labelenumi}{\textbf{\alph{enumi}.}}
%\renewcommand{\labelenumii}{\textbf{\alph{enumii}.}}
%\renewcommand{\labelenumii}{\textbf{(\alph{enumii})}}
\renewcommand{\labelenumii}{\textbf{(\arabic{enumii})}}

\long\def\symbolfootnote[#1]#2{\begingroup\def\thefootnote{\fnsymbol{footnote}}\footnote[#1]{#2}\endgroup}

%%%%%%%%%%%%%%%%%%%%%%%%%%%%
%%%%%%%%%%%%%%%%%%%%%%%%%%%%
%%%%%%%%%%%%%%%%%%%%%%%%%%%%
\usepackage{setspace}
%\singlespacing
%\onehalfspacing
%\doublespacing

\usepackage{comment}
\usepackage{siunitx}
%%%%%%%%%%%%%%%%%%%%%%%%%%%%
\usepackage[x11names]{xcolor} 
%%%%%%%%%%%%%%%%%%%%%%%%%%%%
\usepackage{relsize}
%\newcommand\CC{C\nolinebreak\hspace{-.05em}\raisebox{.4ex}{\relsize{-3}{\textbf{+}}}\nolinebreak\hspace{-.10em}\raisebox{.4ex}{\relsize{-3}{\textbf{+}}}}
\newcommand\CC{C\nolinebreak[4]\hspace{-.05em}\raisebox{.4ex}{\relsize{-3}{\textbf{++}}}}

%\usepackage{showframe}
\usepackage{pgf,tikz}       % - Elementos gráficos
\usepackage{tikzpagenodes}
\usetikzlibrary{backgrounds,calc,positioning}
\usetikzlibrary{positioning, shapes}

\pgfdeclarelayer{background}
\pgfdeclarelayer{foreground}
\pgfsetlayers{background,main,foreground}

\usepackage[
    backend=biber,
    style=numeric-comp,
    language=auto,
    natbib=false,
    block=space,
    isbn=false,
    url=true, 
    doi=true,
    mcite=true,
    eprint=false,
    sorting=none,
    maxnames=6
]{biblatex} % https://ctan.org/pkg/biblatex
\addbibresource{bibliografia.bib}
